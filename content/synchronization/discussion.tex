%!TEX root = ../../thesis.tex
\section{Discussion}
\label{sync-discussion}

All four discussed algorithms and techniques meet the required features that were described in (\refchapter{sync-features}). Their main differences lie in the fulfillment of the user experience requirements.

Locking does not allow users to edit the entire document simultaneously and is therefore unsuitable for the purpose of the music editor. When it comes to composing synthesizer arrangements, collaborative editing provides an advantage because both users can immediately correct mistakes or improve the harmony of a melody.

Although three-way merging does permit users to simultaneously edit the entire document, it does have the drawback of confronting users with merge conflict dialogs and as well as having a lack of real-time updates whilst users are editing themselves. A collaborative work on synthesizer arrangements could lead to many merge conflicts and in addition, users would not see other users' changes whilst editing.

Operational Transformation not only complies with all required features but also complies with the User Experience requirements. Nonetheless, it still has some drawbacks. Firstly, it is difficult to gain an overview of all available algorithms, techniques and their disadvantages because the topic has been researched for so many years, which makes finding the right combination of solutions really hard. Secondly, once decided upon which techniques to use, implementing a failsafe OT system from scratch would take up a great deal of time. Joseph Gentle, former Google Wave\footnote{Google Wave, now known as Apache Wave, is a collaborative multimedia communication tool which used OT to enable real-time editing} engineer, said: ``Wave took 2 years to write and if we rewrote it today, it would take almost as long to write a second time.''\footnote{\cite{gentle2011sharejs}}. Lastly, all papers were only using text documents and their basic operations (e.g. insert and delete) to describe their algorithms. The editor that will result from this thesis will be based on JSON objects. Thus, a large set of new operations would need to be created and tested. The amount of work that is required for this would be a thesis in itself.

Differential Synchronization also complies with both required features and user experience requirements but is, in comparison to OT, much easier to implement as it can be based on existing diff- and patch algorithms. Once again, the original paper only presented solutions for dealing with text documents but the algorithm works independently from the underlying data structures. The only limitation is that the data structure needs to have a suitable diff and patch implementation (which will be shown in \refchapter{impl-sync-algorithm}). All aspects considered, DS appears to be very well-suited for the purpose of a collaborative music editor and is therefore chosen in favor of OT. Currently there are no implementations of this approach where both server and client are written in JavaScript and the base data structure is a JSON document. The implementation of the music editor based on DS in JavaScript would advance the research on DS in general.