%!TEX root = ../../thesis.tex
\chapter{Evaluation}
\label{ch:evaluation}

The resulting prototypical application of this thesis has shown, that it is indeed possible to build a high-performance DAW in modern web technology. Once the initial performance bottlenecks were found and solutions were built to overcome these problems, the editor could be used just like all other desktop DAWs without performance problems. The algorithms that were developed in this thesis (e.g. the scheduler) can be used universally and they are not limited to apply only for web-based DAWs. The editor however, only provides a fraction of the functionality of contemporary DAWs like the ones that were shown in \refchapter{part-concept} and it is not sure that there might not be further performance problems when advanced features are added in future versions of the editor. But there were always solutions to the problems that were described in the previous chapters, there will also be solutions for upcoming problems.

The four audio modules that have been developed for the editor compete well with the functionality that is provided by other DAWs. As already mentioned before, the editor's audio modules are still very basic and other DAWs provide much more detailed and feature-rich modules. For example, other drum machines give in-depth settings for rhythm, volume of individual sounds and allow to create custom drum kits. Whereas, this editor's drum machine only provides a subset of these features. It still works in the same way as other drum machines and produces precise drum beats. Modern DAW applications can look back onto years of experience and research and they built their user interfaces with that knowledge to perfectly suit the needs of musicians. The Web Audio editor still has to catch up with that knowledge to create a competing user experience.

One advantage of the web-based DAW is its synchronization feature. Only a few DAWs provide such functionality and they are often limited to only specific parts of the application. This limitation does not apply to the here developed editor. All aspects of the editor are synced between all clients and no restrictions in collaborative actions apply. It could be shown that the Differential Synchronization algorithm can work well together with the here needed JSON data structure and the algorithm could be enhanced to fit the needs of a truly real-time application. Through their `natural' connectedness, web applications are much easier to synchronize than desktop applications, which is also the most important advantage that \cite[p. 21]{wyse2013viability} pointed out in their review of the Web Audio API. Another advantage of a web-based DAW is that it is automatically available for more than just one platform without the need of special code that interfaces OS-specific APIs.

The biggest disadvantage of a web-based solution is that it needs to be connected at all  time to synchronize and to save a project. This means that musicians are limited to work in environments that have a stable Internet connection. Creating music is a creative process and ideas can spark in many situations and they need to be tested out immediately, so the environment should not interfere with the creative process. The current implementation of the editor however, does this by requiring an Internet connection. This means that the editor cannot be used on planes or on long journeys on the road (e.g. on a tour). The online-restriction addresses a general problem of modern web applications and a lot of research is done in this field for example, by the `Hoodie'\footnote{\url{http://hood.ie/}, accessed on 26.03.2014} development team, which is working on an application agnostic online-offline synchronization system.